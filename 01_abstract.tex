
\keywords{%
  класс типов, 
  Kotlin, 
  полиморфизм
}

\abstractcontent{
Объектно-ориентированные языки программирования существенно изменились за последние несколько десятилетий. Они стали более мощными, более выразительными и, в соответствии с современными тенденциями, более \emph{функциональными}. Сегодня большинство языков программирования семейства Java предоставляют возможность использования таких особенностей функциональных языков программирования, как, например, лямбда-нотация, ленивые вычисления и функции высшего порядка. Тем не менее, некоторые по-настоящему мощные механизмы функциональной парадигмы присутствуют лишь в нескольких объектно-ориентированных языках программирования. Одним из них является \emph{механизм классов типов}. В рамках данной работы будет представлена реализация данного механизма в языке программирования Kotlin. Разработанный прототип является полнофункциональным расширением компилятора и позволяет утверждать, что подобный механизм может существенным образом повысить выразительность объектно-ориентированных языков программирования.      
}

\keywordsen{
  type class,
  Kotlin,
  polymorphism
}

\abstractcontenten{
Over the past decades object-oriented languages changed a lot. They became more expressive, powerful and, as the recent trends dictate, more \emph{functional}. As a result, today most JVM languages feature lambda notation, lazy streams and all the fancy functional stuff which makes program code look better. Of course, these are not the only things functional programming has to offer, and a lot more is very much hidden from mainstream developer’s eyes. One of such features is the \emph{type class pattern}. In this paper an implementation of type classes mechanism in Kotlin will be presented. It is a full-featured prototype built upon Kotlin compiler which allows to argue that such a feature (maybe not of a particular presented design) would be a good choice for a mainstream OOP language seeking to enhance and improve user experience.}