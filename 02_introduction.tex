\intro

%Объектно-ориентированные языки программирования существенно изменились за последние несколько десятилетий. Они стали более мощными, более выразительными и, в соответствии с современными тенденциями, более \emph{функциональными}. Сегодня большинство языков программирования семейства Java предоставляют возможность использования таких особенностей функциональных языков программирования, как, например, лямбда-нотация, ленивые вычисления и функции высшего порядка. Тем не менее, некоторые по-настоящему мощные механизмы функциональной парадигмы присутствуют лишь в нескольких объектно-ориентированных языках программирования. Одним из них является \emph{механизм классов типов}.

Классы типов впервые появились в языке программирования Haskell как альтернативный подход к реализации механизмов \emph{специального полиморфизма} (\emph{ad hoc polymorphism}). В отличие от перегрузок функций, классы типов описывают операции относительно некоторого набора типовых переменных, что позволяет использовать их внутри параметрически полиморфных функций. Каждому кортежу типов, входящему в класс типов, ставится в соответствие собственный словарь функций, который может быть использован внутри параметрически полиморфной функции в качестве носителя операций, переопределенных над этим кортежем типов. При этом конкретный словарь функций выбирается статическим образом на основании значений только лишь типовых переменных, указанных в точке вызова параметрически полиморфной функции. Таким образом, механизм работы классов типов также отличен и от концепции полиморфизма наследования, в котором словарь функций не может быть обособлен от переменных. 

%Классы типов являются особой формой \emph{специального полиморфизма} (\emph{ad hoc polymorphism}). В отличие от механизма перегрузки операторов, типичного для объектно-ориентированных языков программирования, классы типов позволяют определить набор операций над конечным множеством типов отдельно от определения этих типов, что позволяет существенно расширить область применимости специального полиморфизма. Отдельное определение типов и полиморфных операций над ними является здесь существенным, поскольку в контексте объектно-ориентированных языков программирования такое разделение позволяет расширять функциональность существующих иерархий типов без модификации самой иерархии.   

Целью данной работы является разработка и реализация механизма классов типов в языке программирования Kotlin. Несмотря на то, что в качестве эталонного было принято решение, представленное в языке программирования Haskell, были рассмотрены и другие языки программирования, которые предоставляют возможность использования классов типов, что позволило выработать наиболее удобный с точки зрения пользователя подход к выражению новых семантических конструкций языка. %На примере простейших задач будет показано, насколько сильно повышается выразительность языка при использовании классов типов, не приводя при этом к значительному усложнению программного кода.

Остальная часть работы организована следующим образом. В первом разделе будет представлен обзор реализаций концепции классов типов в нескольких языках программирования. Затем, проанализировав преимущества и недостатки рассмотренных решений, в разделе $2$ будут сформулированы требования, которым должен удовлетворять разрабатываемый в рамках данной работы механизм. Четвертый раздел полностью посвящен деталям процесса разработки и реализации. В последнем, пятом разделе будет проведен анализ разработанного механизма классов типов и представлена методика его тестирования. В заключении рассмотрим планы по улучшению полученного решения.        