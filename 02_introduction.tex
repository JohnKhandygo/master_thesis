%%%%%%%%%%%%%%%%%%%%%%%%%%%%%%%%%%%%%%%%%%%%%%%%%%%%%%%%%%%%%%%%%%%%%%%%%%%%%%%%
\intro
%%%%%%%%%%%%%%%%%%%%%%%%%%%%%%%%%%%%%%%%%%%%%%%%%%%%%%%%%%%%%%%%%%%%%%%%%%%%%%%%

Объектно-ориентированные языки программирования претерпевали значительные изменения на протяжении последних десятилетий. Они прибавили в мощности, стали более выразительными и, в соответствии с современными тенденциями, более \emph{функциональными}. Сегодня большинство языков программирования, относящихся к семейству Java, предоставляют пользователю возможности для использования лямбда нотации, ленивых потоков данных, а также некоторых других механизмов, позаимствованных из функционального программирования. Тем не менее, текущие <<функциональные>> возможности объектно-ориентированных языков программирования существенно ограничены, скрывая по-настоящему мощные особенности функциональной парадигмы от глаз рядовых разработчиков. Одной из таких особенностей являются \emph{классы типов}.

В своей основе классы типов являются не более чем реализацией механизмов функционирования \emph{специального полиморфизма} (\emph{ad hoc polymorphism}). В отличие от типичных для объектно-ориентированного программирования \emph{интерфейсов} (\emph{interface}), классы типов предоставляют возможность описывать реализации полиморфных методов классов для конкретных типов отдельно от описания самого класса. Такой подход позволяет существенно расширить область применимости специального полиморфизма. 

В данной работе рассматривается реализация механизма классов типов в языке программирования Kotlin. 
%Здесь стоит отметить, что для Kotlin поддержка подобного расширения является большим подспорьем, поскольку усиливает его позиции относительно другого видного конкурента Java --- Scala, однако для других языков программирования механизм классов типов также может быть хорошим выбором для дальнейшего развития. 
Kotlin изначально позиционировался как более гибкая альтернатива Java, которая в то же время проще для изучения и понимания, нежели Scala. Предлагаемая к рассмотрению реализация учитывает данную особенность философии Kotlin, позволяя существенно усилить его позиции относительно Scala с точки зрения пользователей. Стоит также отметить, что механизм классов типов также может являться хорошим выбором для дальнейшего развития и многих других языков программирования.

Несмотря на то, что в качестве эталонной была принята реализация, представленная в Haskell, были рассмотрены и другие языки программирования, в которых присутствует механизм классов типов. По большей части это было сделано для того, чтобы выработать наиболее удобный с точки зрения пользователя подход к выражению новых семантических конструкций языка. На примере простейших задач будет показано, насколько сильно повышается выразительность языка при использовании данного расширения, в то же время не усложняя программный код.