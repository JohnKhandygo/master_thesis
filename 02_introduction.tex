\intro

Объектно-ориентированные языки программирования существенно изменились за последние несколько десятилетий. Они стали более мощными, более выразительными и, в соответствии с современными тенденциями, более \emph{функциональными}. Сегодня большинство языков программирования семейства Java предоставляют возможность использования таких особенностей функциональных языков программирования, как, например, лямбда-нотация, ленивые вычисления и функции высшего порядка. Тем не менее, некоторые по-настоящему мощные механизмы функциональной парадигмы присутствуют лишь в нескольких объектно-ориентированных языках программирования. Одним из них является \emph{механизм классов типов}.

Классы типов являются особой формой \emph{специального полиморфизма} (\emph{ad hoc polymorphism}). В отличие от механизма перегрузки операторов, типичного для объектно-ориентированных языков программирования, классы типов позволяют определить набор операций над конечным множеством типов отдельно от определения этих типов, что позволяет существенно расширить область применимости специального полиморфизма. Отдельное определение типов и полиморфных операций над ними является здесь существенным, поскольку в контексте объектно-ориентированных языков программирования такое разделение позволяет расширять функциональность существующих иерархий типов без модификации самой иерархии.   

Целью данной работы является разработка и реализация механизма, обеспечивающего функциональность классов типов в языке программирования Kotlin. Несмотря на то, что в качестве эталонной было принято решение, представленная в языке программирования Haskell, были рассмотрены и другие языки программирования, которые предоставляют возможность использования классов типов, что позволило выработать наиболее удобный с точки зрения пользователя подход к выражению новых семантических конструкций языка. На примере простейших задач будет показано, насколько сильно повышается выразительность языка при использовании классов типов, не приводя при этом к значительному усложнению программного кода.

%Остальная часть работы организована следующим образом. Сначала будет представлен обзор реализации механизма классов типов в нескольких языках программирования. Затем перейдем к формулированию требований, которым должен    