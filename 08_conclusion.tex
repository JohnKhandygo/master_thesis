\conclusion

В рамках данной работы был разработан и реализован механизм классов типов в языке программирования Kotlin. Классы типов, так же как и механизм переопределения функций, служат для обеспечения функционирования специального полиморфизма, однако обладают большей выразительностью и гибкостью в смысле управления зависимостями. В ходе работы были рассмотрены реализации концепции классов типов в других языках программирования, включая функциональные и объектно-ориентированные языки. Несмотря на то, что в объектно-ориентированных языках программирования изначально не предусматривалось понятие класса типов, вся необходимая семантика может быть перенесена на существующие синтаксические конструкции интуитивно понятным образом. 

Разработанное решение обладает всей базовой функциональностью, присущей классам типов. В сравнении с подходом, реализованным в языке программирования Scala, полученное решение обладает большей наглядностью и, в силу гарантирования свойства уникальности экземпляров классов типов, проще для понимания. Тем не менее, некоторые аспекты работы реализованного механизма требуют доработки и, возможно, более точного тестирования. В рамках продолжения данной работы представляется целесообразным рассмотреть следующие пути развития представленного решения:
\begin{itemize}
    \item Расширить доступные способы инстанцирования экземпляров классов типов. В рамках данного расширения могут быть рассмотрены не только функции, создающие экземпляры классов типов динамическим образом, но также и особый случай автоматического преобразования переменных к экземплярам классов типов, конструктор которых имеет единственный аргумент.  
    \item Интегрировать механизм, позволяющий описывать ограничения на принадлежность типовых переменных классам типов не только в функциях, но также и при определении классов. 
    \item Добавить возможность описания экземпляров классов типов в виде классов. Такая модификация позволит рассматривать экземпляры классов типов для обобщенных значений типовых переменных. 
    \item Рассмотреть стратегии ограничения области поиска экземпляров классов типов. 
\end{itemize}
Данная работа также может послужить примером того, как концепция классов типов  может быть интегрирована в объектно-ориентированный язык программирования. 