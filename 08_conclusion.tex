\conclusion

В рамках данной работы был разработан и реализован механизм классов типов в языке программирования Kotlin. Классы типов, так же как и перегрузки функций, служат для обеспечения функционирования специального полиморфизма, однако обладают большей выразительностью и гибкостью в смысле управления зависимостями. В ходе работы были рассмотрены реализации концепции классов типов в других языках программирования, включая функциональные и объектно-ориентированные языки. Несмотря на то, что в объектно-ориентированных языках программирования изначально не предусматривалось понятие класса типов, вся необходимая семантика может быть перенесена на существующие синтаксические конструкции интуитивно понятным образом. Более того, данная концепция может использоваться вместе с механизмом перегрузки функций. 

Разработанное решение обладает всей необходимой функциональностью, присущей классам типов. В сравнении с подходом, реализованным в языке программирования Scala, полученное решение обладает большей наглядностью и, в силу гарантирования свойства уникальности экземпляров классов типов, проще для понимания. Тем не менее, некоторые аспекты работы реализованного механизма требуют доработки и, возможно, более точного тестирования.